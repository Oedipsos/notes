\documentclass{report}

\usepackage[margin=3cm]{geometry}
\usepackage[T1]{fontenc}
\usepackage[utf8]{inputenc}
\usepackage[francais]{babel}
\usepackage{lmodern} % For 'xfrac' fonts
\usepackage{amssymb}
\usepackage{amsmath}
\usepackage{amsthm}
\usepackage{mathtools}
\usepackage{xfrac}
\usepackage{empheq}
\usepackage{framed}
\usepackage{graphicx}
\usepackage{xcolor}

\newcommand{\RNum}[1]{\uppercase\expandafter{\romannumeral #1\relax}}
\newcommand{\Nepsilon}{N_\varepsilon}
\newcommand{\reels}{\mathbb{R}}
\newcommand{\naturals}{\mathbb{N}}
\newcommand{\rationals}{\mathbb{Q}}
\newcommand{\complex}{\mathbb{C}}

%\newcounter{thmcount}[subsection]

\makeatletter
\newcases{function}
  {\quad}
  {\hfil$\m@th\displaystyle{##}$\hfil}
  {\hfil$\m@th\displaystyle{##}$\hfil}
  {(}{)}
\makeatother

\theoremstyle{definition}
\newtheorem{definition}{Définition}
\newtheorem*{demo}{Démonstration}
\newtheorem*{example}{Exemple}
\newtheorem*{examples}{Exemples}
\newtheorem*{exercice}{Exercice}

\theoremstyle{plain}
\newtheorem{property}{Propriété}
\newtheorem{theorem}{Théorème}

\theoremstyle{remark}
\newtheorem*{remark}{Remarque}
\newtheorem*{notation}{Notation}
\newtheorem*{notations}{Notations}



\title{
    MATH-F201 \\
    Calcul différentiel et intégral \RNum{2}
}

\author{
    \underline{Notes de cours}
}

\date{
    Année académique 2019-2020
}

\begin{document}
\maketitle

\chapter{Rappels de topologie métrique}
\section{Notion d'espace métrique}

% Définition d'une distance sur X
\begin{leftbar}
    \begin{definition}
        Soit $X$ un ensemble. on appelle \underline{distance sur $X$} toute
        application $d : X \times X \to \reels^+$ telle que
        \begin{enumerate}
            \item{$\forall x, y \in X,\ d(x, y) = d(y, x)$;\hfill
                \textit{(symétrie)}}
            \item{$\forall x, y, z \in X,\ d(x, z) \le d(x, y) + d(y, z)$;
                \hfill\textit{(inégalité triangulaire)}}
            \item{$\forall x, y \in X,\ (d(x, y) = 0 \iff x = y)$.\hfill
                \textit{(séparation des points)}}
        \end{enumerate}
    \end{definition}
\end{leftbar}

% Définition d'espace métrique
\begin{leftbar}
    \begin{definition}
        On appelle \underline{espace métrique} tout couple $(X, d)$ d'un
        ensemble $X$ et d'une distance $d$ sur $X$.
    \end{definition}
\end{leftbar}

% Définition de limite/convergence d'une suite
\begin{leftbar}
    \begin{definition}
        Une suite $(x_n)_{n\in\naturals} \in X^\naturals$ \underline{converge}
        vers $x \in X$ lorsque \\
        \begin{equation*}
            \forall \varepsilon < 0,\ \exists\Nepsilon\in\naturals,\ 
            \forall n \ge\Nepsilon,\ d(x_n, x) < \varepsilon.
        \end{equation*}
    \end{definition}
\end{leftbar}

% Unicité de la limite d'une suite
\begin{leftbar}
    \begin{property}
        Si $(x_n)_{n\in\naturals} \in X^\naturals$ converge vers $x \in X$ 
        dans l'espace métrique $(X, d)$, alors un tel $x$ est unique, et est
        appelé limite de la suite $(x_n)_{n\in\naturals}$.
    \end{property}
\end{leftbar}

\begin{notations}
        $\quad x = \lim\limits_{n \to +\infty} {x_n}$, 
        $\quad x_n \xrightarrow[n \to +\infty]{d} x$\footnotemark
\end{notations}
\footnotetext{A lire "$x_n$ tend vers x lorsque $n$ tend vers $\infty$
    avec la distance $d$" ??}

\newpage

\section{Notion d'espace vectoriel}

% Définition de norme sur un espace vectoriel  
\begin{leftbar}
    \begin{definition}
        Soit $\mathbb{K}$ un sous-corps, et $E$ un $\mathbb{K}$-ev. On appelle
        \underline{norme sur $E$} toute application $N : E \to \reels^+$ telle
        que
        \begin{enumerate}
            \item{$\forall x\in E,\ \lambda\in\mathbb{K},\ N(\lambda x) = 
                \lambda N(x)$;\hfill\textit{(homogénéité)}}
            \item{$\forall x, y \in E,\ N(x+y) \le N(x) + N(y)$;\hfill
                \textit{(inégalité triangulaire)}}
            \item{$\forall x\in E,\ (N(x) = 0 \iff x = 0)$.\hfill
                \textit{(séparation des points\footnotemark)}}
        \end{enumerate}
    \end{definition}
\end{leftbar}
\footnotetext{caractère \underline{défini} de la norme}

% Définition d'espace vectoriel normé
\begin{leftbar}
    \begin{definition}
        On appelle \underline{espace vectoriel normé} tout couple $(E, N)$
        d'un espace vectoriel $E$ sur un sous-corps $\mathbb{K}$ et d'une
        norme $N$ sur $E$.
    \end{definition}
\end{leftbar}

% Norme ~> Distance
\begin{leftbar}
    \begin{property}
        Si $(E, N)$ est un espace vectoriel normé, alors
        \begin{equation*}
            d_N : \begin{function}
            E\times E \to\reels^+ \\ 
            (x, y)\mapsto N(y-x)
            \end{function} 
        \end{equation*}
        est une distance sur $E$. En
        particulier, $(E, d_N)$ est un espace métrique.
    \end{property}
\end{leftbar}

% Définition de série
\begin{leftbar}
    \begin{definition}
        Soit $(E, N)$ un espace vectoriel normé. Soit 
        $(x_n)_{n\in\naturals} \in X^\naturals$. On appelle \underline{série
        de terme} \underline{général $(x_n)$} la suite $(S_n)_{n\ge0}$ définie
        par
        \begin{equation*}
           S_n \coloneqq\sum\limits_{i=0}^{n} x_i.
        \end{equation*}
        
    \end{definition}
\end{leftbar}

% Convergence d'une série
\begin{exercice}
    Soit $S \in E$. On dit que la série $(S_n)_{n\ge0}$ de terme général
    $(u_n)_{n\ge0} \in E^\naturals$ \underline{converge} vers $S$ lorsque
    \begin{equation*}
        \forall\varepsilon > 0,\ \exists\Nepsilon\in\naturals,\ \forall n\ge
        \Nepsilon,\ \begin{cases}
            d(S_n, S) < \varepsilon \\
            N(S_n - S) < \varepsilon \\
            N(\sum\limits_{i=0}^n u_i - S) < \varepsilon
        \end{cases}
    \end{equation*}
    Dans ce cas, $S$ est unique, et vaut $ S = \sum\limits_{i=0}^{+\infty}
    u_i$.
\end{exercice}

\begin{examples}
    \begin{enumerate}
        \item{$(\reels, |\cdot|)$ est un $\rationals$- et $\reels$-evn.}
        \item{$(\reels^d, \|\cdot\|_2)$ est un $\rationals$- et $\reels$-evn.}
        \item{$(\complex, |\cdot|)$ est un $\rationals$-, $\reels$- et
            $\complex$-evn.}
        \item{$(\reels^d, \|\cdot\|_p)$ est un $\rationals$- et $\reels$-evn.}
        \item{$(\complex^d, \|\cdot\|_p)$ est un $\rationals$-, $\reels$- et
            $\complex$-evn.}
        \item{$\|(x_1,\dots,n_d)\|_\infty\coloneqq
            \max\{|x_j|\ |\ j\in \{1,\dots,d\}\}$ est une norme sur $\reels^d$
            ou $\complex^d$.}
        \newpage
        \item{$l^p(N) \coloneqq \{(u_n)_{n\ge0}\in\complex^\naturals\ |\ 
            \sum\limits_n |u_n|^p < +\infty\}$. \\
            En posant $ \|(u_n)\|_p \coloneqq \big(\sum\limits_n
            |u_n|^p\big)^{\sfrac{1}{p}}$, on a que $(l^p(N), \|\cdot\|_p)$
            est un $\complex$-evn.}
        \item{$\reels[x] = \{\sum\limits_{n=0}^d a_nx^n \text{ avec }
        (a_n) \in \reels^\naturals \text{ presque tous nuls }\}$}
        \item{$([0,1], d(x, y) = |y-x|)$ est un espace métrique.}
    \end{enumerate}
\end{examples}
\newpage

\section{Suites de Cauchy}

\begin{leftbar}
    \begin{definition}
        Une suite $(x_n)_{n\ge0}\in X^\naturals$, où $(X, d)$ est un espace
        métrique, est dite de \underline{Cauchy} lorsque
        \begin{equation*}
            \forall\varepsilon > 0,\ \exists\Nepsilon\in\naturals,\ 
            \forall n, m\ge\Nepsilon,\ d(x_n, x_m) < \varepsilon.
        \end{equation*}
    \end{definition}
\end{leftbar}

\begin{leftbar}
    \begin{property}
        Si $(x_n)_{n\ge0}\in X^\naturals$ converge vers $x\in X$ dans l'espace
        métrique $(X, d)$, alors elle est de Cauchy dans $(X, d)$.
    \end{property}
\end{leftbar}

\begin{demo}{exercice}
\end{demo}

Notons que la réciproque n'est généralement pas vérifiée, d'où la définition
suivante.

\begin{leftbar}
    \begin{definition}
        Soit $(X, d)$ un espace métrique. On dit que $(X, d)$ est
        \underline{complet} lorsque toute suite de Cauchy converge dans
        $(X, d)$.
    \end{definition}
\end{leftbar}

\begin{examples}
    \begin{itemize}
        \item[$*$]{$(\reels, |\cdot|),\ (\reels^d, \|\cdot\|_2),\ 
            (\complex, |\cdot|),\ (\complex^d, \|\cdot\|_2) $ sont
            \textbf{\underline{complets}}.}
        \item[$*$]{$(\reels[x], \|\cdot\|)$ n'est \textbf{
        \underline{pas complet}}.}
    \end{itemize}
\end{examples}

\vspace{1cm}
\section{Continuité}

\begin{leftbar}
    \begin{definition}
        Soient $(X, d_1),\ (Y, d_2)$ deux espaces métriques. Soit
        $f : X \to Y$, et soit $x\in X$. On dit que $f$ est
        \underline{continue en $x$} lorsque $\forall y\in X$, 
        \begin{equation*}
            \forall\varepsilon > 0,\ \exists\delta > 0,\ 
            d_1(x, y) < \delta \Rightarrow d_2(f(x), f(y)) < \varepsilon.
        \end{equation*}
        $f$ est dite \underline{continue sur $X$} lorsqu'elle est continue
        en tout $x\in X$. 
    \end{definition}
\end{leftbar}

\begin{leftbar}
    \begin{property}[Caractérisation séquencielle de la continuité]
        Soient $(X, d_1),\ (Y, d_2)$ deux espaces métriques, $f : X\to Y$
        et $x\in X$. La fonction $f$ est continue en $x$ si et seulement si
        pour toute suite $(x_n)_{n\ge0}\in X^\naturals$ convergent vers $x$,
        la suite $(f(x_n))_{n\ge0} \in Y^\naturals$ converge vers $f(x)$.
    \end{property}
\end{leftbar}

\begin{example}
    Soit $i\in\naturals$. Considérons la fonction 
    \begin{equation*}
        c_i : \begin{function}
            (\reels[x], \|\cdot\|) \to (\reels, |\cdot|) \\
            P = \sum\limits_k a_kx^k \mapsto a_i
        \end{function}.
    \end{equation*}
    Soit $P = \sum\limits_k a_k x^k$ et $Q = \sum\limits_k b_k x^k$ dans
    $\reels[x]$. Observons que $c_i(P-Q) = c_i(\sum\limits_k (a_k-b_k) x^k)
    = a_i-b_i$, donc $|c_i(P) - c_i(Q)| \le \|P-Q\|$.
    Si $P$ est fixé dans $\reels[x]$ et $(Q_n)_{n\in\naturals} \in (\reels[x])
    ^\naturals$ avec $ Q_n \xrightarrow[n\to+\infty]{\|\cdot\|} P$, alors
    $c_i(Q) \xrightarrow[n\to+\infty]{|\cdot|}  c_i(P)$.

    Ainsi, $c_i$ est continue en tout $P\in\reels[x]$.
    Donc $c_i$ est continue sur $\reels[x]$
\end{example}

    Posons pour $n\ge1$, $P_n = \sum\limits_{k=0}^n \frac{1}{k!}x^k$.
    Observons que, pour $n > m \ge 1$, on a
    \begin{align*}
        \|P_n-P_m\| &= \left\|\sum\limits_{k=0}^n \frac{1}{k!}x^k - \sum
        \limits_{k=0}^m \frac{1}{k!}x^k\right\| \\
        &= \left\|\sum\limits_{k=m+1}^n \frac{1}{k!}x^k \right\| \\ 
        &= \frac{1}{(m+1)!}
    \end{align*}
    Avec $\|P\| = \max\limits_{k\in\naturals} |a_k|$. \\
    On en déduit que $(P_n)_{n\ge0}$ est de Cauchy dans $(\reels[x],
    \|\cdot\|)$.
    \paragraph{}
    Raisonnons par l'absurde et supposons qu'il existe $P\in\reels[x]$ tel que
    $P_n \xrightarrow[n\to+\infty]{\|\cdot\|} P$. Alors pour tout $i\in
    \naturals$, on a $c_i(P_n) \xrightarrow[n\to+\infty]{|\cdot|} c_i(P)$ par
    continuité de $c_i$ en $P$.

    Ainsi, $\forall i\in\naturals,\ c_i(P) = \frac{1}{i!}$\hfill(Contradiction
    avec $P\in\reels[x]$ !)
    \paragraph{}
    Donc $(P_n)$ ne converge pas, et $(\reels[x], \|\cdot\|)$ n'est pas
    complet !
\end{example}

\chapter{Suites de fonctions - Convergences}

\end{document}